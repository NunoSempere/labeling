\nonstopmode{}
\documentclass[letterpaper]{book}
\usepackage[times,inconsolata,hyper]{Rd}
\usepackage{makeidx}
\usepackage[utf8,latin1]{inputenc}
% \usepackage{graphicx} % @USE GRAPHICX@
\makeindex{}
\begin{document}
\chapter*{}
\begin{center}
{\textbf{\huge Package `labeling'}}
\par\bigskip{\large \today}
\end{center}
\begin{description}
\raggedright{}
\item[Type]\AsIs{Package}
\item[Title]\AsIs{Axis Labeling}
\item[Version]\AsIs{0.4.3}
\item[Date]\AsIs{2023-08-29}
\item[Author]\AsIs{Justin Talbot,}
\item[Maintainer]\AsIs{Nuno Sempere }\email{nuno.semperelh@gmail.com}\AsIs{}
\item[Description]\AsIs{Functions which provide a range of axis labeling algorithms.}
\item[License]\AsIs{MIT + file LICENSE | Unlimited}
\item[Collate]\AsIs{'labeling.R'}
\item[NeedsCompilation]\AsIs{no}
\item[Imports]\AsIs{stats, graphics}
\end{description}
\Rdcontents{\R{} topics documented:}
\inputencoding{utf8}
\HeaderA{labeling-package}{Axis labeling}{labeling.Rdash.package}
\aliasA{labeling}{labeling-package}{labeling}
\keyword{dplot}{labeling-package}
%
\begin{Description}\relax
Functions for positioning tick labels on axes
\end{Description}
%
\begin{Details}\relax

\Tabular{ll}{ Package: & labeling\\{} Type: &
Package\\{} Version: & 0.4.3\\{} Date: & 2023-08-29\\{}
License: & Unlimited\\{} LazyLoad: & yes\\{} }

Implements a number of axis labeling schemes, including
those compared in An Extension of Wilkinson's Algorithm
for Positioning Tick Labels on Axes by Talbot, Lin, and
Hanrahan, InfoVis 2010.
\end{Details}
%
\begin{Author}\relax
Justin Talbot \email{justintalbot@gmail.com}
\end{Author}
%
\begin{References}\relax
Heckbert, P. S. (1990) Nice numbers for graph labels,
Graphics Gems I, Academic Press Professional, Inc.
Wilkinson, L. (2005) The Grammar of Graphics,
Springer-Verlag New York, Inc. Talbot, J., Lin, S.,
Hanrahan, P. (2010) An Extension of Wilkinson's Algorithm
for Positioning Tick Labels on Axes, InfoVis 2010.
\end{References}
%
\begin{SeeAlso}\relax
\code{\LinkA{extended}{extended}}, \code{\LinkA{wilkinson}{wilkinson}},
\code{\LinkA{heckbert}{heckbert}}, \code{\LinkA{rpretty}{rpretty}},
\code{\LinkA{gnuplot}{gnuplot}}, \code{\LinkA{matplotlib}{matplotlib}},
\code{\LinkA{nelder}{nelder}}, \code{\LinkA{sparks}{sparks}},
\code{\LinkA{thayer}{thayer}}, \code{\LinkA{pretty}{pretty}}
\end{SeeAlso}
%
\begin{Examples}
\begin{ExampleCode}
heckbert(8.1, 14.1, 4)	# 5 10 15
wilkinson(8.1, 14.1, 4)	# 8 9 10 11 12 13 14 15
extended(8.1, 14.1, 4)	# 8 10 12 14
# When plotting, extend the plot range to include the labeling
# Should probably have a helper function to make this easier
data(iris)
x <- iris$Sepal.Width
y <- iris$Sepal.Length
xl <- extended(min(x), max(x), 6)
yl <- extended(min(y), max(y), 6)
plot(x, y,
    xlim=c(min(x,xl),max(x,xl)),
    ylim=c(min(y,yl),max(y,yl)),
    axes=FALSE, main="Extended labeling")
axis(1, at=xl)
axis(2, at=yl)
\end{ExampleCode}
\end{Examples}
\inputencoding{utf8}
\HeaderA{extended}{An Extension of Wilkinson's Algorithm for Position Tick Labels on Axes}{extended}
%
\begin{Description}\relax
\code{extended} is an enhanced version of Wilkinson's
optimization-based axis labeling approach. It is
described in detail in our paper. See the references.
\end{Description}
%
\begin{Usage}
\begin{verbatim}
  extended(dmin, dmax, m, Q = c(1, 5, 2, 2.5, 4, 3),
    only.loose = FALSE, w = c(0.25, 0.2, 0.5, 0.05))
\end{verbatim}
\end{Usage}
%
\begin{Arguments}
\begin{ldescription}
\item[\code{dmin}] minimum of the data range

\item[\code{dmax}] maximum of the data range

\item[\code{m}] number of axis labels

\item[\code{Q}] set of nice numbers

\item[\code{only.loose}] if true, the extreme labels will be
outside the data range

\item[\code{w}] weights applied to the four optimization
components (simplicity, coverage, density, and
legibility)
\end{ldescription}
\end{Arguments}
%
\begin{Value}
vector of axis label locations
\end{Value}
%
\begin{Author}\relax
Justin Talbot \email{justintalbot@gmail.com}
\end{Author}
%
\begin{References}\relax
Talbot, J., Lin, S., Hanrahan, P. (2010) An Extension of
Wilkinson's Algorithm for Positioning Tick Labels on
Axes, InfoVis 2010.
\end{References}
\inputencoding{utf8}
\HeaderA{extended.figures}{Generate figures from An Extension of Wilkinson's Algorithm for Position Tick Labels on Axes}{extended.figures}
%
\begin{Description}\relax
Generates Figures 2 and 3 from our paper.
\end{Description}
%
\begin{Usage}
\begin{verbatim}
  extended.figures(samples = 100)
\end{verbatim}
\end{Usage}
%
\begin{Arguments}
\begin{ldescription}
\item[\code{samples}] number of samples to use (in the paper we
used 10000, but that takes awhile to run).
\end{ldescription}
\end{Arguments}
%
\begin{Value}
produces plots as a side effect
\end{Value}
%
\begin{Author}\relax
Justin Talbot \email{justintalbot@gmail.com}
\end{Author}
%
\begin{References}\relax
Talbot, J., Lin, S., Hanrahan, P. (2010) An Extension of
Wilkinson's Algorithm for Positioning Tick Labels on
Axes, InfoVis 2010.
\end{References}
\inputencoding{utf8}
\HeaderA{gnuplot}{gnuplot's labeling algorithm}{gnuplot}
%
\begin{Description}\relax
gnuplot's labeling algorithm
\end{Description}
%
\begin{Usage}
\begin{verbatim}
  gnuplot(dmin, dmax, m)
\end{verbatim}
\end{Usage}
%
\begin{Arguments}
\begin{ldescription}
\item[\code{dmin}] minimum of the data range

\item[\code{dmax}] maximum of the data range

\item[\code{m}] number of axis labels
\end{ldescription}
\end{Arguments}
%
\begin{Value}
vector of axis label locations
\end{Value}
%
\begin{Author}\relax
Justin Talbot \email{justintalbot@gmail.com}
\end{Author}
%
\begin{References}\relax
\url{http://www.gnuplot.info/}
\end{References}
\inputencoding{utf8}
\HeaderA{heckbert}{Heckbert's labeling algorithm}{heckbert}
%
\begin{Description}\relax
Heckbert's labeling algorithm
\end{Description}
%
\begin{Usage}
\begin{verbatim}
  heckbert(dmin, dmax, m)
\end{verbatim}
\end{Usage}
%
\begin{Arguments}
\begin{ldescription}
\item[\code{dmin}] minimum of the data range

\item[\code{dmax}] maximum of the data range

\item[\code{m}] number of axis labels
\end{ldescription}
\end{Arguments}
%
\begin{Value}
vector of axis label locations
\end{Value}
%
\begin{Author}\relax
Justin Talbot \email{justintalbot@gmail.com}
\end{Author}
%
\begin{References}\relax
Heckbert, P. S. (1990) Nice numbers for graph labels,
Graphics Gems I, Academic Press Professional, Inc.
\end{References}
\inputencoding{utf8}
\HeaderA{matplotlib}{Matplotlib's labeling algorithm}{matplotlib}
%
\begin{Description}\relax
Matplotlib's labeling algorithm
\end{Description}
%
\begin{Usage}
\begin{verbatim}
  matplotlib(dmin, dmax, m)
\end{verbatim}
\end{Usage}
%
\begin{Arguments}
\begin{ldescription}
\item[\code{dmin}] minimum of the data range

\item[\code{dmax}] maximum of the data range

\item[\code{m}] number of axis labels
\end{ldescription}
\end{Arguments}
%
\begin{Value}
vector of axis label locations
\end{Value}
%
\begin{Author}\relax
Justin Talbot \email{justintalbot@gmail.com}
\end{Author}
%
\begin{References}\relax
\url{https://matplotlib.org/}
\end{References}
\inputencoding{utf8}
\HeaderA{nelder}{Nelder's labeling algorithm}{nelder}
%
\begin{Description}\relax
Nelder's labeling algorithm
\end{Description}
%
\begin{Usage}
\begin{verbatim}
  nelder(dmin, dmax, m,
    Q = c(1, 1.2, 1.6, 2, 2.5, 3, 4, 5, 6, 8, 10))
\end{verbatim}
\end{Usage}
%
\begin{Arguments}
\begin{ldescription}
\item[\code{dmin}] minimum of the data range

\item[\code{dmax}] maximum of the data range

\item[\code{m}] number of axis labels

\item[\code{Q}] set of nice numbers
\end{ldescription}
\end{Arguments}
%
\begin{Value}
vector of axis label locations
\end{Value}
%
\begin{Author}\relax
Justin Talbot \email{justintalbot@gmail.com}
\end{Author}
%
\begin{References}\relax
Nelder, J. A. (1976) AS 96. A Simple Algorithm for
Scaling Graphs, Journal of the Royal Statistical Society.
Series C., pp. 94-96.
\end{References}
\inputencoding{utf8}
\HeaderA{rpretty}{R's pretty algorithm implemented in R}{rpretty}
%
\begin{Description}\relax
R's pretty algorithm implemented in R
\end{Description}
%
\begin{Usage}
\begin{verbatim}
  rpretty(dmin, dmax, m = 6, n = floor(m) - 1,
    min.n = n%/%3, shrink.sml = 0.75, high.u.bias = 1.5,
    u5.bias = 0.5 + 1.5 * high.u.bias)
\end{verbatim}
\end{Usage}
%
\begin{Arguments}
\begin{ldescription}
\item[\code{dmin}] minimum of the data range

\item[\code{dmax}] maximum of the data range

\item[\code{m}] number of axis labels

\item[\code{n}] number of axis intervals (specify one of
\code{m} or \code{n})

\item[\code{min.n}] nonnegative integer giving the
\emph{minimal} number of intervals. If \code{min.n == 0},
\code{pretty(.)} may return a single value.

\item[\code{shrink.sml}] positive numeric by a which a default
scale is shrunk in the case when \code{range(x)} is very
small (usually 0).

\item[\code{high.u.bias}] non-negative numeric, typically
\code{> 1}. The interval unit is determined as
\code{\{1,2,5,10\}} times \code{b}, a power of 10. Larger
\code{high.u.bias} values favor larger units.

\item[\code{u5.bias}] non-negative numeric multiplier favoring
factor 5 over 2. Default and 'optimal': \code{u5.bias =
  .5 + 1.5*high.u.bias}.
\end{ldescription}
\end{Arguments}
%
\begin{Value}
vector of axis label locations
\end{Value}
%
\begin{Author}\relax
Justin Talbot \email{justintalbot@gmail.com}
\end{Author}
%
\begin{References}\relax
Becker, R. A., Chambers, J. M. and Wilks, A. R. (1988)
\emph{The New S Language}. Wadsworth \& Brooks/Cole.
\end{References}
\inputencoding{utf8}
\HeaderA{sparks}{Sparks' labeling algorithm}{sparks}
%
\begin{Description}\relax
Sparks' labeling algorithm
\end{Description}
%
\begin{Usage}
\begin{verbatim}
  sparks(dmin, dmax, m)
\end{verbatim}
\end{Usage}
%
\begin{Arguments}
\begin{ldescription}
\item[\code{dmin}] minimum of the data range

\item[\code{dmax}] maximum of the data range

\item[\code{m}] number of axis labels
\end{ldescription}
\end{Arguments}
%
\begin{Value}
vector of axis label locations
\end{Value}
%
\begin{Author}\relax
Justin Talbot \email{justintalbot@gmail.com}
\end{Author}
%
\begin{References}\relax
Sparks, D. N. (1971) AS 44. Scatter Diagram Plotting,
Journal of the Royal Statistical Society. Series C., pp.
327-331.
\end{References}
\inputencoding{utf8}
\HeaderA{thayer}{Thayer and Storer's labeling algorithm}{thayer}
%
\begin{Description}\relax
Thayer and Storer's labeling algorithm
\end{Description}
%
\begin{Usage}
\begin{verbatim}
  thayer(dmin, dmax, m)
\end{verbatim}
\end{Usage}
%
\begin{Arguments}
\begin{ldescription}
\item[\code{dmin}] minimum of the data range

\item[\code{dmax}] maximum of the data range

\item[\code{m}] number of axis labels
\end{ldescription}
\end{Arguments}
%
\begin{Value}
vector of axis label locations
\end{Value}
%
\begin{Author}\relax
Justin Talbot \email{justintalbot@gmail.com}
\end{Author}
%
\begin{References}\relax
Thayer, R. P. and Storer, R. F. (1969) AS 21. Scale
Selection for Computer Plots, Journal of the Royal
Statistical Society. Series C., pp. 206-208.
\end{References}
\inputencoding{utf8}
\HeaderA{wilkinson}{Wilkinson's labeling algorithm}{wilkinson}
%
\begin{Description}\relax
Wilkinson's labeling algorithm
\end{Description}
%
\begin{Usage}
\begin{verbatim}
  wilkinson(dmin, dmax, m,
    Q = c(1, 5, 2, 2.5, 3, 4, 1.5, 7, 6, 8, 9),
    mincoverage = 0.8,
    mrange = max(floor(m/2), 2):ceiling(6 * m))
\end{verbatim}
\end{Usage}
%
\begin{Arguments}
\begin{ldescription}
\item[\code{dmin}] minimum of the data range

\item[\code{dmax}] maximum of the data range

\item[\code{m}] number of axis labels

\item[\code{Q}] set of nice numbers

\item[\code{mincoverage}] minimum ratio between the the data
range and the labeling range, controlling the whitespace
around the labeling (default = 0.8)

\item[\code{mrange}] range of \code{m}, the number of tick
marks, that should be considered in the optimization
search
\end{ldescription}
\end{Arguments}
%
\begin{Value}
vector of axis label locations
\end{Value}
%
\begin{Note}\relax
Ported from Wilkinson's Java implementation with some
changes.  Changes: 1) m (the target number of ticks) is
hard coded in Wilkinson's implementation as 5.  Here we
allow it to vary as a parameter. Since m is fixed,
Wilkinson only searches over a fixed range 4-13 of
possible resulting ticks.  We broadened the search range
to max(floor(m/2),2) to ceiling(6*m), which is a larger
range than Wilkinson considers for 5 and allows us to
vary m, including using non-integer values of m.  2)
Wilkinson's implementation assumes that the scores are
non-negative. But, his revised granularity function can
be extremely negative. We tweaked the code to allow
negative scores.  We found that this produced better
labelings.  3) We added 10 to Q. This seemed to be
necessary to get steps of size 1.  It is possible for
this algorithm to find no solution.  In Wilkinson's
implementation, instead of failing, he returns the
non-nice labels spaced evenly from min to max.  We want
to detect this case, so we return NULL. If this happens,
the search range, mrange, needs to be increased.
\end{Note}
%
\begin{Author}\relax
Justin Talbot \email{justintalbot@gmail.com}
\end{Author}
%
\begin{References}\relax
Wilkinson, L. (2005) The Grammar of Graphics,
Springer-Verlag New York, Inc.
\end{References}
\printindex{}
\end{document}
